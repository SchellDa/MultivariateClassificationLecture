\PassOptionsToPackage{usenames,dvipsnames,svgnames,table}{xcolor}
\documentclass[compress]{beamer}
\usetheme{KSETA}

\usepackage[T1]{fontenc}
\usepackage{CJKutf8}

\usepackage{enumitem}
\setitemize{label=\usebeamerfont*{itemize item}%
  \usebeamercolor[fg]{itemize item}
    \usebeamertemplate{itemize item}}

\usepackage{minted}
\newminted{cpp}{gobble=4,linenos}

\usepackage{graphics}
\graphicspath{{images/}}

\usepackage{pgf}
\usepackage{pgffor}
\usepackage{tikz}
\usepackage{pgfplots}
\usepgfplotslibrary{colormaps,fillbetween}
%\pgfplotsset{compat=1.14}
\usetikzlibrary{pgfplots.colorbrewer,pgfplots.fillbetween}

\usetikzlibrary{shadows.blur}
\usetikzlibrary{shapes.symbols}
\usetikzlibrary{calc,intersections,matrix}

\usepackage{ifthen}
\usetikzlibrary{arrows,fadings,automata,snakes,shapes,shapes.misc,shapes.arrows,trees,positioning,calc,decorations.pathreplacing,arrows.meta}
\usepackage[binary-units=true]{siunitx}
\usepackage{wasysym}
\usepackage{pgfplots}
\usepackage{listings}
\usepackage{xcolor}

\usepackage{tcolorbox}
\usepackage{booktabs}

\usepackage[framemethod=TikZ]{mdframed}
\mdfdefinestyle{simplebox}{roundcorner=4pt,linewidth=0,backgroundcolor=blue!50!black,fontcolor=white}

\usepackage{multicol}
\usepackage{etoolbox}

\makeatletter
\patchcmd{\beamer@sectionintoc}{\vskip1.5em}{\vskip1.0em}{}{}
\makeatother

\AtBeginSection[] {
  \begin{frame}
    \frametitle{\insertsection}
  \vspace{-0.5cm}
  \begin{center}
  \begin{scriptsize}
  \begin{multicols}{2}
    \tableofcontents[sectionstyle=show/shaded,subsectionstyle=show/show/hide]
    \end{multicols}
  \end{scriptsize}
  \end{center}
  \end{frame}
}


\tikzfading[name=arrowfading, top color=transparent!0, bottom color=transparent!95]
\tikzset{arrowfill/.style={top color=blue!50, bottom color=blue,}}
\tikzset{arrowstyle/.style={draw=black,arrowfill, single arrow,minimum height=#1, single arrow,
single arrow head extend=.4cm,}}

\tikzset{
    box/.style={
      rectangle,
	  color=#1,
      draw=black,
      fill=#1,
      thick,
      text=black,
      align=center,
      rounded corners=6pt,
      blur shadow={shadow blur steps=5},
      minimum height=1.5em
    }, 
    hbox/.style={
      rectangle,
      draw=black,
      fill=black,
      thick,
      text=white,
      align=center,
      rounded corners=6pt,
      blur shadow={shadow blur steps=5},
      minimum height=1.5em
    }, 
}

\colorlet{tree0}{blue}
\colorlet{tree100}{white!20!blue}
\colorlet{tree200}{white!40!blue}
\colorlet{tree300}{white!60!blue}
\colorlet{tree400}{white!80!blue}
\colorlet{tree500}{white}
\colorlet{tree600}{white!80!orange}
\colorlet{tree700}{white!60!orange}
\colorlet{tree800}{white!40!orange}
\colorlet{tree900}{white!20!orange}
\colorlet{tree1000}{orange}

%--- Tango-Colors definieren -------------------------------------------
\definecolor{Tblue}{HTML}{3465A4}	% tango sky blue 2
\definecolor{Tbluedark}{HTML}{204A87}	% tango sky blue 3
\definecolor{Tbluelight}{HTML}{729FCF}	% tango sky blue 1
\definecolor{Tbrown}{HTML}{C17D11}	% tango chocolate 2
\definecolor{Tbrowndark}{HTML}{8F5902}	% tango chocolate 3
\definecolor{Tbrownlight}{HTML}{E9B96E}	% tango chocolate 1
\definecolor{Tgray}{HTML}{888A85}	% tango aluminium 4
\definecolor{Tgraydark}{HTML}{555753}	% tango aluminium 5
\definecolor{Tgraydarker}{HTML}{2E3436}	% tango aluminium 5
\definecolor{Tgraylight}{HTML}{BABDB6}	% tango aluminium 3
\definecolor{Tgraylight2}{HTML}{E4E6E2}	% Sehr hell (für Tabellenköpfe)
\definecolor{Tgraylight3}{HTML}{F0F2EE}	% Sehr hell (für Quelltexte)
\definecolor{Tgreen}{HTML}{73D216}	% tango chameleon 2	
\definecolor{Tgreendark}{HTML}{4E9A06}	% tango chameleon 3
\definecolor{Tgreenlight}{HTML}{8AE234}	% tango chameleon 1
\definecolor{Tred}{HTML}{CC0000}	% tango scarlet red 2
\definecolor{Treddark}{HTML}{A40000}	% tango scarlet red 3
\definecolor{Tredlight}{HTML}{EF2929}	% tango scarlet red 1	
\definecolor{Tlilac}{HTML}{75507B}	% tango plum 2
\definecolor{Tlilacdark}{HTML}{5C3566}	% tango plum 3
\definecolor{Tlilaclight}{HTML}{AD7FA8}	% tango plum 1
\definecolor{Tyellow}{HTML}{EDD400}	% tango butter 2
\definecolor{Tyellowdark}{HTML}{C4A000}	% tango butter 1
\definecolor{Tyellowlight}{HTML}{FCE94F}% tango butter 3
\definecolor{Torange}{HTML}{F57900}	% tango orange 2
\definecolor{Torangedark}{HTML}{CE5C00}	% tango orange 1
\definecolor{Torangelight}{HTML}{FCAF3E}% tango orange 3
